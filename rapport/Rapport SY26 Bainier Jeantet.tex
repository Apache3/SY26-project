\documentclass[a4paper,12pt]{article}
\author{Baptiste Bainier et Thomas Jeantet}
\usepackage[french]{babel}
\usepackage{amsmath}
\usepackage{graphicx}
\usepackage{amsfonts}
\usepackage{pdflscape}
\usepackage[utf8]{inputenc}
\usepackage{float}



%Package
\usepackage[margin=1in]{geometry}
\usepackage{fancyhdr}
\usepackage{placeins}
\usepackage{listings}
\usepackage{color}
\usepackage[table,xcdraw]{xcolor}
\usepackage{ulem} %barrer du texte
\usepackage{cancel}% barrer dans une expression math (\cancel{})
\usepackage{pgf,tikz}
\usepackage{mathrsfs}
\usepackage{multirow}
%\usepackage{gensymb}
\usepackage{caption}
\usepackage{eurosym}% pour le symbole €



\usetikzlibrary{shapes.geometric, arrows}
\definecolor{qqqqff}{rgb}{0.,0.,1.}
%Configuration
\renewcommand*\contentsname{Sommaire}
\graphicspath{ {images/} }
%\renewcommand{\thesection}{\Roman{section}}
%\renewcommand{\thesubsection}{\Alph{subsection}}

\definecolor{codegreen}{rgb}{0,0.6,0}
\definecolor{codegray}{rgb}{0.5,0.5,0.5}
\definecolor{codepurple}{rgb}{0.58,0,0.82}
\definecolor{backcolour}{rgb}{0.95,0.95,0.92}
 
\lstdefinestyle{mystyle}{
    backgroundcolor=\color{backcolour},   
    commentstyle=\color{codegreen},
    keywordstyle=\color{magenta},
    numberstyle=\tiny\color{codegray},
    stringstyle=\color{codepurple},
    basicstyle=\footnotesize,
    breakatwhitespace=false,         
    breaklines=true,                 
    captionpos=b,                    
    keepspaces=true,                 
    numbers=left,                    
    numbersep=5pt,                  
    showspaces=false,                
    showstringspaces=false,
    showtabs=false,                  
    tabsize=2
}

\lstset{style=mystyle}
\renewcommand{\lstlistingname}{Script}


\pagestyle{fancy}
\fancyhf{}
\rhead{Baptiste Bainier et Thomas Jeantet}
\lhead{SY26 - Détection de formes par réseau de neurones}
\rfoot{Page \thepag}

\title{Projet SY26\\Détection de formes par réseau de neurones}
%\graphicspath{}
\begin{document}
\maketitle
\newpage
\tableofcontents
\newpage


\section*{Introduction}
  - BlaBla

\section{Choix du framework}
  - Solution tout faire en C++ avec openCV
  - Ou trouver un framework adapté : Caffe

\section{Prise en main de Caffe}
  - Présentation du framework
  - Réalisation des exemples (MNIST)

\section{Adaptation du framework}
  - Création des images de train (data augmentation)
  - Création de nos propres databases compatibles (format lmdb)
  - Adaptation du net : taille des images, format...

\section{Evaluation des performances}
  - influence du format des images
  - influence des images de train (database sale ou pas)
  - influence apprentissage : taille de batch, itérations...

\section*{Conclusion}
  - Comment améliorer le projet
  - La technologie du machine learning
  - Ce que le projet nous a apporté

\end{document}
