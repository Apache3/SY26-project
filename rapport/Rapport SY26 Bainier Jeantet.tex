\documentclass[a4paper,12pt]{article}
\author{Baptiste Bainier et Thomas Jeantet}
\usepackage[french]{babel}
\usepackage{amsmath}
\usepackage{graphicx}
\usepackage{amsfonts}
\usepackage{pdflscape}
\usepackage[utf8]{inputenc}
\usepackage{float}



%Package
\usepackage[margin=1in]{geometry}
\usepackage{fancyhdr}
\usepackage{placeins}
\usepackage{listings}
\usepackage{color}
\usepackage[table,xcdraw]{xcolor}
\usepackage{ulem} %barrer du texte
\usepackage{cancel}% barrer dans une expression math (\cancel{})
\usepackage{pgf,tikz}
\usepackage{mathrsfs}
\usepackage{multirow}
%\usepackage{gensymb}
\usepackage{caption}
\usepackage{eurosym}% pour le symbole €



\usetikzlibrary{shapes.geometric, arrows}
\definecolor{qqqqff}{rgb}{0.,0.,1.}
%Configuration
\renewcommand*\contentsname{Sommaire}
\graphicspath{ {images/} }
%\renewcommand{\thesection}{\Roman{section}}
%\renewcommand{\thesubsection}{\Alph{subsection}}

\definecolor{codegreen}{rgb}{0,0.6,0}
\definecolor{codegray}{rgb}{0.5,0.5,0.5}
\definecolor{codepurple}{rgb}{0.58,0,0.82}
\definecolor{backcolour}{rgb}{0.95,0.95,0.92}
 
\lstdefinestyle{mystyle}{
    backgroundcolor=\color{backcolour},   
    commentstyle=\color{codegreen},
    keywordstyle=\color{magenta},
    numberstyle=\tiny\color{codegray},
    stringstyle=\color{codepurple},
    basicstyle=\footnotesize,
    breakatwhitespace=false,         
    breaklines=true,                 
    captionpos=b,                    
    keepspaces=true,                 
    numbers=left,                    
    numbersep=5pt,                  
    showspaces=false,                
    showstringspaces=false,
    showtabs=false,                  
    tabsize=2
}

\lstset{style=mystyle}
\renewcommand{\lstlistingname}{Script}


\pagestyle{fancy}
\fancyhf{}
\rhead{Baptiste Bainier et Thomas Jeantet}
\lhead{SY26 - Détection de formes par réseau de neurones}
\rfoot{Page \thepage}

\title{Projet SY26\\Détection de formes par réseau de neurones}
%\graphicspath{}
\begin{document}
\maketitle
\newpage
\newpage


\section*{Introduction}
  Le développement de notre réseau de neurones a été implémenté dans le cadre de l'UV SY26. L'objectif est de pouvoir détecter et différencier des formes apparaissant sur un écran. Il y a 6 formes différentes, chacunes de couleur différente. Il faut pouvoir les différencier quelle que soit leur taille et leur position sur l'écran.

  Le réseau doit être implémenté sur Raspberry Pi 3, et les images sont capturées par une Raspicam (v2).
\bigskip
\tableofcontents

\newpage
\section{Choix du framework}
  Dans un premier temps, nous avons du nous accorder sur le choix de l'environnement de travail. Deux options nous semblaient évidentes : 
  \begin{itemize}
    \item Programmer en C++ avec OpenCV
    \item Programmer à l'aide d'un framework
  \end{itemize}
  Nous avons étudié les avantages et les inconvénients de chaque méthode avant de commencer la programmation.
  
  \subsection{Utiliser OpenCV}
    La première option évidente aurait été de dévlopper nous même tout le programme en C++ à l'aide des librairies OpenCV. Il existse quelques librairies OpenCV permettant d'implémenter des réseaux de neurones, mais ces librairies n'utilisent que des coiches MLP, ce qui est donc moins adapté que la framework que nous allons utiliser pour le traitement d'images.

    De plus, nous avons estimé que développer tout le code nous même en C++ avec OpenCV nous aurait pris énormément de temps, ce qui, couplé aux technologies restreintes de OpenCV (MLP uniquement), en fait une méthode peu intéressante.
  
  \subsection{Utiliser un framework}
    La deuxième solution est donc d'utiliser un framework. Cette méthode présente également quelques inconvénients : 
    \begin{itemize}
      \item Le temps de recherche des différents frameworks existants
      \item Le temps de formation au framework choisi
    \end{itemize}

    En effet, dans un premier temps nous avons dû nous rensigner sur les différents framework existants. Cette phase demande du temps, car il faut trouver tous les frameworks adaptés à notre sujet, puis trouver le plus adapté selonla documentation de chacun. Plusieurs frameworks se sont révélés intéressants, dont TensorFlow, PyTorch, ou Caffe que nous avons choisi.

    Utiliser un framework présente aussi des avantages :
    \begin{itemize}
      \item Le temps de déploiement du réseau
      \item La flexibilité du réseau
    \end{itemize}

    Le réseau est plus rapide à mettre en place, car il existe déjà de nombreuses "briques" programmées, ce qui nous permet d'avoir à seulement les adapter à notre cas. On dispose également d'un réseau plus flexible, car tous les éléments sont prévus pour que les paramètre puissent être changés facilement, ce qui nous évite de refaire tout le code pour un moindre changement dans le premier scénario.

    Nous avons donc choisi d'utiliser un framework, ce qui nous économise du temps de développement et nous permet de concentrer nos efforts sur l'optimisation du réseau de neurones plutôt que sur le développement de ce dernier. Nous avons choisi d'utiliser Caffe, car il semblait parfaitement adapté à notre projet (réseau de neurone pour la détection d'images), et de plus il est en grande partie rédigé en C++, qui est le langage que nous connaissons le mieux.

\newpage
\section{Prise en main de Caffe}
  Une fois le framework choisi, il faut le prendre en main. Il est nécessaire de comprendre son utilité et son fonctionnement avant de commencer à l'utiliser. Nous allons donc brièvement le présenter dans cette partie.
  
  \subsection{Du Caffe ?}
    \textit{"Caffe is a deep learning framework made with expression, speed, and modularity in mind."}
    Caffe est un framework qui permet de créer des réseaux de neurones sur-mesure, selon les besoins de l'utilisateur. Plusieurs "couches" sont déjà implémentées, il suffit ensuite de choisir quelles couches utiliser, et quels paramètres leur appliquer. Cette étape de design du modèle se fait facilement au travers d'un simple fichier type protobuf, contenant l'ordre et les paramètres de chaque couche. De nombreux modèles différents existent déjà, et sont mis à disposition par Caffe au travers de leur "zoo".

    L'utilisateur peut alors choisir d'utiliser une base de données existante, ou de créer sa propre database. De nombreux exemples sont déjà implémentés dans Caffe, ce qui nous permet de découvrir le fonctionnement du framework, mais aussi de nous inspirer de ces exemples pour créer notre réseau. Dans un premier temps, nous avons découvert le framework grâce à l'exemple du modèle LeNet, utilisé pour différencier les chiffres de la ma base de données MNIST.
  
  \subsection{L'exemple MNIST}
    Caffe fournit un exemple très complet basé sur la classification de la base de données MNIST à l'aide du modèle LeNet. Nous nous sommes basés sur cet exemple pour comprendre le fonctionnement 

\newpage
\section{Adaptation du framework}
  
  \subsection{Data augmentation}
  
  \subsection{Création de database(s)}
  
  \subsection{Personnalisation du net}

\newpage
\section{Evaluation des performances}
  Plusieurs paramètres influencent les performances d'un réseau, en plus de ses différentes couches. Nous allons détailler ici les paramètres que nous avons pu modifier au cours de ce projet, et leurs influences.
  \subsection{Le format des images}
    La formes des images que nous allions utiliser fut le premier paramètre que nous avons déterminé. 
    
    Tout d'abord, il faut choisir entre l'utilisation d'images couleur ou d'images en nuances de gris. Cependant, plus le nombre de canaux est grand, plus le temps d'apprentissage sera long. Nous avons tout de même choisi d'utiliser des images couleur, car chaque forme est associée à une couleur. 

    Il devrait être plus simple pour le réseau de repérer les différences entre chaque image.
    Ensuite il nous faut décider de la taille de chaque image. Une grande image impose beaucoup de calculs, il faut donc choisir une taille qui permette de distinguer les formes, quelle que soit leur taille. Nous avons choisi un format rectangulaire, de 96 par 54 pixel. Cette taille est un multiple de la résolution maximale de la caméra, 1920 par 1080, ce qui permet moins de pertes de qualité lors d'un changement de taille. Elle est également suffisamment grande pour discerner des petites formes.
    
  \subsection{Influence des images de la database}

  
  \subsection{Influence des hyperParamètres}
    L'apprentissage est une étape cruciale, possédant beaucoup de paramètres, dont nous allons voir l'influence.
    \subsubsection{Le learning rate et son évolution}
      Le learning rate est le paramètre le plus important lors de l'apprentissage; delui dépend la convergence ou non d'un réseau. Caffe nous permet de de faire varier ce learning rate en fonction du nombre d'itérations.

      Il n'y a pas de méthode particulière pour déterminer les bons paramètres, il faut lancer des apprentissages et observer la réponse du réseau. A force de tatônnements, nous sommes parvenus à une courbe de learning rate produisant une convergence assez rapide en phase de validation (environ 400 iterations), et permettant un loss bas, et une accuracy proche de 1.

    \subsubsection{Taille de batch}
      Un batch est le nombre d'image que l'on va fournir à notre réseau avant de changer les poids. Plus il y a d'images dans un batch, plus la descente de gradient sera précise, mais plus l'apprentissage est lent. De plus, si la taille de batch n'est pas un multiple du nombre d'images dans la base de donnée, on risque de favoriser l'apprentissage d'une classe au détriment d'une autre. 

      Selon le nombre d'images issus de la data augmentation, nous choisissons les batch size aux alentours de 60. Avecc cet ordre de grandeur, nous avons une vitesse d'apprentissage acceptable tout en convergeant.
    \subsubsection{Nombre de filtres}
      

\newpage
\section*{Conclusion}
  
  \subsection*{Améliorer notre réseau}
  
  \subsection*{Le Machine Learning}
  
  \subsection*{Ce que le projet nous a apporté}

\end{document}
