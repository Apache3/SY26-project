\documentclass[a4paper,12pt]{article}
\author{Baptiste Bainier et Thomas Jeantet}
\usepackage[french]{babel}
\usepackage{amsmath}
\usepackage{graphicx}
\usepackage{amsfonts}
\usepackage{pdflscape}
\usepackage[utf8]{inputenc}
\usepackage{float}



%Package
\usepackage[margin=1in]{geometry}
\usepackage{fancyhdr}
\usepackage{placeins}
\usepackage{listings}
\usepackage{color}
\usepackage[table,xcdraw]{xcolor}
\usepackage{ulem} %barrer du texte
\usepackage{cancel}% barrer dans une expression math (\cancel{})
\usepackage{pgf,tikz}
\usepackage{mathrsfs}
\usepackage{multirow}
%\usepackage{gensymb}
\usepackage{caption}
\usepackage{eurosym}% pour le symbole €



\usetikzlibrary{shapes.geometric, arrows}
\definecolor{qqqqff}{rgb}{0.,0.,1.}
%Configuration
\renewcommand*\contentsname{Sommaire}
\graphicspath{ {images/} }
%\renewcommand{\thesection}{\Roman{section}}
%\renewcommand{\thesubsection}{\Alph{subsection}}

\definecolor{codegreen}{rgb}{0,0.6,0}
\definecolor{codegray}{rgb}{0.5,0.5,0.5}
\definecolor{codepurple}{rgb}{0.58,0,0.82}
\definecolor{backcolour}{rgb}{0.95,0.95,0.92}
 
\lstdefinestyle{mystyle}{
    backgroundcolor=\color{backcolour},   
    commentstyle=\color{codegreen},
    keywordstyle=\color{magenta},
    numberstyle=\tiny\color{codegray},
    stringstyle=\color{codepurple},
    basicstyle=\footnotesize,
    breakatwhitespace=false,         
    breaklines=true,                 
    captionpos=b,                    
    keepspaces=true,                 
    numbers=left,                    
    numbersep=5pt,                  
    showspaces=false,                
    showstringspaces=false,
    showtabs=false,                  
    tabsize=2
}

\lstset{style=mystyle}
\renewcommand{\lstlistingname}{Script}


\pagestyle{fancy}
\fancyhf{}
\rhead{Baptiste Bainier et Thomas Jeantet}
\lhead{SY26 - Détection de formes par réseau de neurones}
\rfoot{Page \thepage}

\title{Projet SY26\\Détection de formes par réseau de neurones}
%\graphicspath{}
\begin{document}
\maketitle
\newpage
\tableofcontents
\newpage


\section*{Introduction}
  Le développement de notre réseau de neurones a été implémenté dans le cadre de l'UV SY26. L'objectif est de pouvoir détecter et différencier des formes apparaissant sur un écran. Il y a 6 formes différentes, chacunes de couleur différente. Il faut pouvoir les différencier quelle que soit leur taille et leur position sur l'écran.
  \\Le réseau doit être implémenté sur Raspberry Pi 3, et les images sont capturées par une Raspicam (v2).
\section*{Choix du framework}
  \subsection{Utiliser OpenCV}
  \subsection{Utiliser un framework}

\section*{Prise en main de Caffe}
  \subsection{Du Caffe ?}
  \subsection{L'exemple MNIST}

\section*{Adaptation du framework}
  \subsection{Data augmentation}
  \subsection{Création de database(s)}
  \subsection{Personnalisation du net}

\section*{Evaluation des performances}
  \subsection{Influence du format des images}
  \subsection{Influence des images de la database}
  \subsection{Influence des hyperParamètres}

\section*{Conclusion}
  \subsection{Améliorer notre réseau}
  \subsection{Le Machine Learning}
  \subsection{Ce que le projet nous a apporté}

\end{document}
